\input{Algo1Macros}
\usepackage{caratula}
\usepackage{enumerate}
\usepackage{hyperref}
\usepackage{graphicx}
\usepackage{amsfonts}
\usepackage{enumitem}

\decimalpoint
\hypersetup{colorlinks=true, linkcolor=black, urlcolor=blue}
\setlength{\parindent}{0em}
\setlength{\parskip}{0.5em}
\setcounter{tocdepth}{2} % profundidad de indice
\setcounter{section}{0} % nro de section
\renewcommand{\thesubsubsection}{\thesubsection.\Alph{subsubsection}}
\graphicspath{ {images/} }

\everymath{\displaystyle}

% End latex config

\begin{document}
	\titulo{Primer Parcial (Haskell)}
	\fecha{2do cuatrimestre 2023}
	\materia{Algoritmos y Estructuras de Datos I - Introducción a la Programación}
	\integrante{Fausto N. Martínez}{363/23}{fnmartinez@dc.uba.ar}
	
	%Carátula
	\maketitle
	\newpage
	
	\section{Viva la democracia}
	La elección periódica de los gobernantes es la base de los Estados Modernos. Este sistema, denominado "democracia" (término proveniente de la antigua Grecia), tiene diferentes variaciones, que incluyen diferentes formas de elección del/a máximo/a mandatario/a. Por ejemplo, en algunos países se eligen representantes en un colegio electoral (EEUU). En otros se vota a los/as miembros del parlamento (España). En nuestro país elegimos de forma directa la fórmula presidencial (Presidente/a y Vicepresidente/a) cada 4 años.
	
	A continuación presentamos una serie de ejercicios que tienen como objetivo implementar funciones para sistema de escrutinio de una elección presidencial. Leer las descripciones y especificaciones e implementar las funciones requeridas en Haskell, utilizado sóĺamente las herramientas vistas en clase.
	
	Las fórmulas presidenciales serán representadas por tuplas (String x String), donde la primera componente será el nombre del candidato a presidente, y la segunda componente será el nombre del candidato a vicepresidente.
	
	En los problemas en los cuales se reciban como parámetro secuencias de fórmulas y votos, cada posición de la lista votos representará la cantidad de votos obtenidos por la fórmula del parámetro formulas en esa misma posición. Por ejemplo, si la lista de fórmulas es [("Juan Pérez","Susana García"), ("María Montero","Pablo Moreno")] y la lista de votos fuera [34, 56], eso indicaría que la fórmula encabezada por María Montero obtuvo 56 votos, y la lista encabezada por Juan Pérez obtuvo 34 votos.
	
	\subsection{Ejercicio 1 - Votos en Blanco}
	$\texttt{problema votosEnBlanco}(formulas:seq<String \times String>,votos:seq<\mathbb{Z}>,cantTotalVotos:\mathbb{Z}) : \mathbb{Z} \lbrace$\\
	$\texttt{requiere}:\lbrace formulasValidas(formulas) \rbrace$\\
	$\texttt{requiere}:\lbrace |formulas| = |votos| \rbrace$\\
	$\texttt{requiere}:\lbrace$ Todos los elementos de $votos$ son mayores o iguales que 0$\rbrace$\\
	$\texttt{requiere}:\lbrace$ La suma de todos los elementos de $votos$ es menor o igual a $cantTotalVotos \rbrace$\\
	$\texttt{asegura}:\lbrace res$ es la cantidad de votos emitidos que no correspondieron a niguna de las fórmulas que se presentaron $\rbrace$\\
	$\rbrace$
	
	\subsection{Ejercicio 2 - Fórmulas Válidas}
	$\texttt{problema formulasValidas}(formulas:seq<String \times String>) : Bool \lbrace$\\
	$\texttt{requiere}:\lbrace True \rbrace$\\
	$\texttt{asegura}:\lbrace (res=true)\Iff formulas$  no contiene nombres repetidos, es decir que cada candidato está en una única fórmula (no se puede ser candidato a presidente y a vicepresidente ni en la misma fórmula ni en fórmulas distintas) $\rbrace$\\
	$\rbrace$
	
	\subsection{Ejercicio 3 - Porcentaje de Votos}
	$\texttt{problema porcentajeDeVotos}(presidente:String,formulas:seq<String \times String>,votos:seq<\mathbb{Z}>) : \mathbb{R} \lbrace$\\
	$\texttt{requiere}:\lbrace $La primera componente de algun elemento de $formulas$ es $presidente \rbrace$\\
	$\texttt{requiere}:\lbrace formulasValidas(formulas) \rbrace$\\
	$\texttt{requiere}:\lbrace |formulas| = |votos| \rbrace$\\
	$\texttt{requiere}:\lbrace$ Todos los elementos de $votos$ son mayores o iguales que 0$\rbrace$\\
	$\texttt{requiere}:\lbrace$ Hay al menos un elemento de $votos$ que es mayor estricto que 0$\rbrace$\\
	$\texttt{asegura}:\lbrace res$ es el porcentaje de votos que obtuvo la fórmula encabezada por presidente sobre el total de votos afirmativos $\rbrace$\\
	$\rbrace$
	
	Para resolver este ejercicio pueden utilizar la siguiente función que devuelve como Float la división entre dos números de tipo Int:
	
	$division :: Int \rightarrow Int \rightarrow Float\\
	division$ $a$ $b$ = $(fromIntegral$ $a)$ / $(fromIntegral$ $b)$
	\newpage
	\subsection{Ejercicio 4 - Próximo Presidente}
	$\texttt{problema proximoPresidente}(formulas:seq<String \times String>,votos:seq<\mathbb{Z}>) : String \lbrace$\\
	$\texttt{requiere}:\lbrace $La primera componente de algun elemento de $formulas$ es $presidente \rbrace$\\
	$\texttt{requiere}:\lbrace formulasValidas(formulas) \rbrace$\\
	$\texttt{requiere}:\lbrace |formulas| = |votos| \rbrace$\\
	$\texttt{requiere}:\lbrace$ Todos los elementos de $votos$ son mayores o iguales que 0$\rbrace$\\
	$\texttt{requiere}:\lbrace$ Hay al menos un elemento de $votos$ que es mayor estricto que 0$\rbrace$\\
	$\texttt{requiere}:\lbrace |formulas|>0\rbrace$\\
	$\texttt{asegura}:\lbrace res$ es el candidato a presidente de $formulas$ más votado de acuerdo a los votos contabilizados en $votos \rbrace$\\
	$\rbrace$
	
	
\end{document}