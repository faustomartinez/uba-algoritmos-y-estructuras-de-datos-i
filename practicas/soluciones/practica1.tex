\input{Algo1Macros}
\usepackage{caratula}
\usepackage{enumerate}
\usepackage{hyperref}
\usepackage{graphicx}
\usepackage{amsfonts}
\usepackage{enumitem}

\decimalpoint
\hypersetup{colorlinks=true, linkcolor=black, urlcolor=blue}
\setlength{\parindent}{0em}
\setlength{\parskip}{0.5em}
\setcounter{tocdepth}{2} % profundidad de indice
\setcounter{section}{0} % nro de section
\renewcommand{\thesubsubsection}{\thesubsection.\Alph{subsubsection}}
\graphicspath{ {images/} }

% End latex config

\begin{document}
	
	\titulo{Práctica 1}
	\fecha{2do cuatrimestre 2023 }
	\materia{Algoritmos y Estructuras de Datos I / Introducción a la Programación}
	\integrante{Fausto N. Martínez}{363/23}{fnmartinez@dc.uba.ar}
	
	%Carátula
	\maketitle
	\newpage
	
	%Indice
	\tableofcontents
	\newpage
	
	% Aca empieza lo propio del documento
	\section{Práctica 1}
	
	\subsection{Ejercicio 1}
	
	Me piden determinar si dados p y q variables preposicionales, las expresiones son \emph{formulas bien formadas}.\\
	$\bigstar$ Rdo.: una formula está bien formada si cumple: 
	
	\includegraphics[width=\textwidth]{FBF}
	
	\subsubsection{Pregunta A}
	\begin{enumerate}[label=(\alph*)]
		\item $(p \neg q)$ no es una fórmula bien formada.
		\item $p \vee q \wedge True$ no es una fórmula bien formada pues da lugar a ambigüedad por la falta de paréntesis.
		\item $p \vee q \wedge True$ no es una fórmula bien formada pues da lugar a ambigüedad por la falta de paréntesis.
		\item $\neg (p)$ no es una fórmula bien formada pues el paréntesis es redundante.
		\item $(p \vee \neg q \wedge q)$ no es una fórmula bien formada ya que la falta de paréntesis da lugar a ambigüedad.
		\item $(True \wedge True \wedge True)$ es una formula bien formada.
		\item $(\neg p)$ no es una formula bien formada ya que no hacen falta los paréntesis.
		\item $(p \vee False)$ es una formula bien formada.
		\item $(p = q)$ es una formula bien formada.
	\end{enumerate}
	
	\subsection{Ejercicio 2}
	\begin{enumerate}[label=(\alph*)]
		\item Bien definida
		\item Bien definida
		\item Mal definida. El conector lógico $\vee$ solo acepta variables del tipo Bool pero x e y son $\mathbb{Z}$ 
		\item Bien definida
		\item Mal definida. $(z = 0)$ y $(z = 1)$ no tipa correctamente dado que z es de tipo Bool.
		\item Mal definida. No tipa correctamente dado que $(y < 0)$ es de tipo Bool y la suma solo acepta números.
	\end{enumerate}
	
	\subsection{Ejercicio 3}
	Primero se evalúa $ \alpha = (3+7 = \pi - 8)$ que al ser una igualdad devuelve un valor del tipo Bool. Luego $\alpha \in \{True, False\}$
	y la fórmula resulta $\alpha \wedge True$ que está bien formada.
	
	\subsection{Ejercicio 4}
	Se que $a = True$, $b = True$, $c = True$, $x = False$, $y = False$
	
	\begin{enumerate}[label=(\alph*)]
		\item ($\neg$True $\vee$ True) = False $\vee$ True = True
		\item (True $\vee$ (False $\wedge$ False) $\vee$ True) = (True $\vee$ True $\vee$ True) = True
		\item $\neg$(True $\vee$ False) = $\neg$True = False
		\item ($\neg$(True $\vee$ False)\Iff($\neg$True $\wedge$ $\neg$False)) = ($\neg$True\Iff(False $\wedge$ True)) = (False\Iff False) = True
		\item ((True $\vee$ False)$\wedge$(False $\vee$ True)) = (True $\wedge$ True) = True
		\item ((True $\vee$ False)$\wedge$(False $\vee$ True))\Iff(True $\vee$ (False $\wedge$ False) $\vee$ True), por (e) y (b) respectivamente, cada uno de estos vale True, entonces tenemos True\Iff True = True
		\item ($\neg$ True $\wedge$ $\neg$ False) = (False $\wedge$ True) = False
	\end{enumerate}
	
	Ahora, si $a = False$, $b = False$, $c = False$, $x = True$, $y = True$
	
	\begin{enumerate}[label=(\alph*)]
		\item ($\neg$False $\vee$ False) = True $\vee$ False = True
		\item (False $\vee$ (True $\wedge$ True) $\vee$ False) = (False $\vee$ True $\vee$ False) = True
		\item $\neg$(False $\vee$ True) = $\neg$True = False
		\item ($\neg$(False $\vee$ True)\Iff($\neg$False $\wedge$ $\neg$True)) = ($\neg$True\Iff(True $\wedge$ False)) = (False\Iff False) = True
		\item ((False $\vee$ True)$\wedge$(True $\vee$ False)) = (True $\wedge$ True) = True
		\item ((False $\vee$ True)$\wedge$(True $\vee$ False))\Iff(False $\vee$ (True $\wedge$ True) $\vee$ False), por (d) y (b) respectivamente, cada uno de estos vale True, entonces tenemos True\Iff True = True
		\item ($\neg$ False $\wedge$ $\neg$ True) = (True $\wedge$ False) = False
	\end{enumerate}
	
	\subsection{Ejercicio 5}
	$\bigstar$ Rdo.: Ua fórmula es \textbf{tautología} si siempre toma el valor True, es \textbf{contradicción} si siempre toma el valor False,
	es \textbf{contingencia} si no es ni tautología ni contradicción.
	
	\subsubsection{Inciso A}
	\begin{tabular}{c|c}
		p & (p $\vee$ $\neg$p) \\
		V & V               \\
		V & V               \\
		F & V               \\
		F & V
	\end{tabular}
	
	Es una tautología
	
	\subsubsection{Inciso B}
	\begin{tabular}{c|c}
		p & (p $\wedge$ $\neg$p) \\
		V & F               \\
		F & F
	\end{tabular}
	
	Es una contradicción
	
	\subsubsection{Inciso C}
	\begin{tabular}{c|c|c|c|c}
		p & q & ($\neg$p $\vee$ q) & (p $\rightarrow$ q) & (($\neg$p $\vee$ q) \Iff (p $\rightarrow$ q)) \\
		V & V & V                  & V                   & V                                              \\
		V & F & F                  & F                   & V                                              \\
		F & V & V                  & V                   & V                                              \\
		F & F & V                  & V                   & V                                              \\
	\end{tabular}
	
	Es una tautología. Observar que esto demuestra (p $\rightarrow$ q) = ($\neg$p $\vee$ q) (La Implicación Material)
	
	\subsubsection{Inciso D}
	\begin{tabular}{c|c|c|c}
		p & q & (p $\vee$ q) & ((p $\vee$ q) $\rightarrow$ p) \\
		V & V & V              & V \\
		V & F & V              & V \\
		F & V & V              & F \\
		F & F & F              & V
	\end{tabular}
	
	Es una contingencia.
	
	\subsubsection{Inciso E}
	
	Sean $\alpha$ = $\neg$(p $\wedge$ q); $\beta$ = ($\neg$p $\vee$ $\neg$q)
	
	\begin{tabular}{c|c|c|c|c}
		p & q & $\alpha$ & $\beta$ & $(\alpha \Iff \beta)$\\
		V & V & F & F & V \\
		V & F & V & V & V \\ 
		F & V & V & V & V \\
		F & F & V & V & V 
	\end{tabular}
	
	Es una tautología. Observar que esto demuestra $\neg$(p $\wedge$ q) = ($\neg$p $\vee$ $\neg$q) (La Ley de De Morgan para la conjunción) 
	
	\subsubsection{Inciso F}
	\begin{tabular}{c|c|c|c|c|c|c}
		p & q & $\neg$p & $\neg$q & ($\neg$p $\wedge$ q) & ($\neg$p $\wedge$ $\neg$ q) & (($\neg$p $\wedge$ q)\Iff($\neg$p $\wedge$ $\neg$ q)) \\
		V & V & F & F & F & F & V \\
		V & F & F & V & F & V & F \\
		F & V & V & F & V & V & V \\
		F & F & V & V & F & V & F \\
	\end{tabular}
	
	Es una contingencia.
	
	\subsubsection{Inciso G}
	\begin{tabular}{c|c}
		p & (p $\rightarrow$ p) \\
		V & V\\
		F & V
	\end{tabular}
	
	Es una tautología.
	
	\subsubsection{Inciso H}
	
	\begin{tabular}{c|c|c|c}
		p & q & (p$\wedge$q) & ((p$\wedge$q)$\rightarrow$p) \\
		V & V & V & V \\
		V & F & F & V \\
		F & V & F & V \\
		F & F & F & V \\
	\end{tabular}
	
	Es una tautología.
	
	\subsubsection{Inciso I}
	
	\begin{tabular}{c|c|c|c|c|c|c|c|c}
		p & q & r & (q$\vee$r) & (p $\wedge$ (q$\vee$r)) & (p$\wedge$q) & (p$\wedge$r) & ((p$\wedge$q)$\vee$(p$\wedge$r)) & ((p $\wedge$ (q$\vee$r))\Iff((p$\wedge$q)$\vee$(p$\wedge$r)))\\
		V & V & V & V & V & V & V & V & V \\
		V & V & F & V & V & V & F & V & V \\
		V & F & V & V & V & F & V & V & V \\
		V & F & F & F & F & F & F & F & V \\
		F & V & V & V & F & F & F & F & V \\
		F & V & F & V & F & F & F & F & V \\
		F & F & V & V & F & F & F & F & V \\
		F & F & F & F & F & F & F & F & V \\
	\end{tabular}
	
	Es una tautología. Observar que esto demuestra que p$\wedge$(q$\vee$r) = (p$\wedge$q)$\vee$(p$\wedge$r) (La propiedad distributiva para la conjunción)
	
	\subsubsection{Inciso J}
	
	Sean $\alpha = (q\rightarrow r)$ ; $\beta = (p \rightarrow q)$ ; $\sigma = (p \rightarrow r)$
	
	\begin{tabular}{c|c|c|c|c|c|c|c|c}
		p & q & r & $\alpha$ & $(p \rightarrow \alpha)$ & $\beta$ & $\sigma$ & $(\beta \rightarrow \sigma)$ & $((p \rightarrow \alpha) \rightarrow (\beta \rightarrow \sigma))$\\
		V & V & V & V & V & V & V & V & V \\
		V & V & F & F & F & V & F & F & V \\
		V & F & V & V & V & F & V & V & V \\
		V & F & F & V & V & F & F & V & V \\
		F & V & V & V & V & V & V & V & V \\
		F & V & F & F & V & V & V & V & V \\
		F & F & V & V & V & V & V & V & V \\
		F & F & F & V & V & V & V & V & V
	\end{tabular}
	
	Es una tautología.
	
	\subsection{Ejercicio 6}
	\begin{enumerate}[label=(\alph*)]
		\item $\emph{False}$ es más fuerte que $\emph{True}$.
		\item $(p \wedge q)$ es más fuerte que $(p \vee q)$.
		\item $\emph{True}$ es más fuerte que $\emph{True}$.
		\item $(p \wedge q)$ es más fuerte que $p$.
		\item $\emph{False}$ es más fuerte que $\emph{False}$.
		\item $p$ es más fuerte que $(p\vee q)$.
		\item No hay relación de fuerza.
		\item No hay relación de fuerza.
	\end{enumerate}
	
	\subsection{Ejercicio 7}
	$\emph{False}$ es la m\'as fuerte,  $\emph{True}$ la m\'as d\'ebil
	\subsection{Ejercicio 8}
	\begin{enumerate}[label=(\alph*)]
		\item 
		\begin{tabular}{c|c}
			p & (False $\wedge$ p)\\
			V & F\\
			F & F\\
		\end{tabular}
		(False $\wedge$ p)$\equiv$False (Dominaci\'on de la conjunci\'on)
		\item 
		\begin{tabular}{c|c}
			p & (True $\vee$ p)\\
			V & V\\
			F & V\\
		\end{tabular}
		(True $\vee$ p)$\equiv$False (Dominaci\'on de la disyunci\'on)
		\item 
		\begin{tabular}{c|c}
			p & (True $\wedge$ p)\\
			V & V\\
			F & F\\
		\end{tabular}
		(True $\wedge$ p)$\equiv$p (Neutro de la conjunci\'on)
		\item 
		\begin{tabular}{c|c}
			p & (False $\vee$ p)\\
			V & V\\
			F & V\\
		\end{tabular}
		(True $\vee$ p)$\equiv$p (Neutro de la disyunci\'on)
		\item 
		\begin{tabular}{c|c}
			p & (p $\wedge$ p)\\
			V & V\\
			F & F\\
		\end{tabular}
		(p $\wedge$ p)$\equiv$p (Idempotencia de la conjunci\'on)
		\item 
		\begin{tabular}{c|c}
			p & (p $\vee$ p)\\
			V & V\\
			F & F\\
		\end{tabular}
		(p $\vee$ p)$\equiv$p (Idempotencia de la disyunci\'on)
		\item 
		\begin{tabular}{c|c|c}
			p & $\neg$p &(p $\vee$ $\neg$p)\\
			V & F & V\\
			F & V & V\\
		\end{tabular}
		(p $\vee$ $\neg$p)$\equiv$True (Inversa de la disyunci\'on)
		\item 
		\begin{tabular}{c|c|c}
			p & $\neg$p &(p $\wedge$ $\neg$p)\\
			V & F & F\\
			F & V & F\\
		\end{tabular}
		(p $\wedge$ $\neg$p)$\equiv$False (Inversa de la conjunci\'on)
		\item 
		\begin{tabular}{c|c|c}
			p & $\neg$p & $\neg$$\neg$p\\
			V & F & V\\
			F & V & F\\
		\end{tabular}
		$\neg$$\neg$p$\equiv$p (Doble negaci\'on)
		\item 
		\begin{tabular}{c|c|c|c}
			p & q & (p$\vee$q) & (p$\wedge$(p$\vee$q))\\
			V & V & V & V\\
			V & F & V & V\\
			F & V & V & F\\
			F & F & F & F\\
		\end{tabular}
		(p$\wedge$(p$\vee$q))$\equiv$p (Absorción de la conjunción)
		\item 
		\begin{tabular}{c|c|c|c}
			p & q & (p$\wedge$q) & (p$\vee$(p$\wedge$q))\\
			V & V & V & V\\
			V & F & F & V\\
			F & V & F & F\\
			F & F & F & F\\
		\end{tabular}
		(p$\vee$(p$\wedge$q))$\equiv$p (Absorción de la disyunción)
		\item Probado en 5.E
		$\neg$(p$\wedge$q)$\equiv$($\neg$p$\wedge$$\neg$q) (Ley de De Morgan para la conjunción)
		\item 
		\begin{tabular}{c|c|c|c|c|c|c}
			p & q & (p$\vee$q) & $\neg$(p$\vee$q) & $\neg$p & $\neg$q & ($\neg$p$\wedge$$\neg$q)\\
			V & V & V & F & F & F & F \\
			V & F & V & F & F & V & F \\
			F & V & V & F & V & F & F \\
			F & F & F & V & V & V & V \\
		\end{tabular}
		$\neg$(p$\vee$q)$\equiv$($\neg$p$\wedge$$\neg$q) (Ley de De Morgan para la disyunción)
		\item
		(p$\wedge$q)$\equiv$(q$\wedge$p) (Conmutatividad para la conjunción)
		\item
		(p$\vee$q)$\equiv$(q$\vee$p) (Conmutatividad para la disyunción)
		\item 
		\begin{tabular}{c|c|c|c|c|c|c}
			p & q & r & (q $\wedge$ r) & (p$\wedge$(q $\wedge$ r)) & (p$\wedge$q) & ((p$\wedge$q)$\wedge$r)\\
			V & V & V & V & V & V & V \\
			V & V & F & F & F & V & F \\
			V & F & V & F & F & F & F \\
			V & F & F & F & F & F & F \\
			F & V & V & V & F & F & F \\
			F & V & F & F & F & F & F \\
			F & F & V & F & F & F & F \\
			F & F & F & F & F & F & F \\
		\end{tabular}
		(p$\wedge$(q $\wedge$ r))$\equiv$((p$\wedge$q)$\wedge$r) (Asociatividad de la conjunción)
		\item 
		\begin{tabular}{c|c|c|c|c|c|c}
			p & q & r & (q $\vee$ r) & (p$\vee$(q $\vee$ r)) & (p$\vee$q) & ((p$\vee$q)$\vee$r)\\
			V & V & V & V & V & V & V \\
			V & V & F & V & V & V & V \\
			V & F & V & V & V & V & V \\
			V & F & F & F & V & V & V \\
			F & V & V & V & V & V & V \\
			F & V & F & V & V & V & V \\
			F & F & V & V & V & F & V \\
			F & F & F & F & F & F & F \\
		\end{tabular}
		(p$\vee$(q $\vee$ r))$\equiv$((p$\vee$q)$\vee$r) (Asociatividad de la disyunción)
		\item Probado en 5.I (Distributividad de la conjunción)
		\item 
		\begin{tabular}{c|c|c|c|c|c|c|c}
			p & q & r & (q $\wedge$ r) & (p$\vee$(q $\wedge$ r)) & (p$\vee$q) &  (p$\vee$r) & ((p$\vee$q)$\wedge$(p$\vee$r))\\
			V & V & V & V & V & V & V & V\\
			V & V & F & F & V & V & V & V\\
			V & F & V & F & V & V & V & V\\
			V & F & F & F & V & V & V & V\\
			F & V & V & V & V & V & V & V\\
			F & V & F & F & F & V & F & F\\
			F & F & V & F & F & F & V & F\\
			F & F & F & F & F & F & F & F\\
		\end{tabular}\\
		(p$\vee$(q $\wedge$ r))$\equiv$((p$\vee$q)$\wedge$(p$\vee$r)) (Distibutividad de la disyunción)
		\item 
		\begin{tabular}{c|c|c|c|c|c}
			p & q & (p$\rightarrow$q) & $\neg$p & $\neg$q & ($\neg$q$\rightarrow$$\neg$p)\\
			V & V & V & F & F & V\\
			V & F & F & F & V & F\\
			F & V & V & V & F & V\\
			F & F & V & V & V & V\\
		\end{tabular}
		(p$\rightarrow$q)$\equiv$($\neg$q$\rightarrow$$\neg$p) (Contraposición lógica)
		\item Probado en 5.C (Implicación Material)
		\item 
		\begin{tabular}{c|c|c|c|c|c}
			p & q & (p\Iff q) & (p$\rightarrow$q) & (q$\rightarrow$p) & ((p$\rightarrow$q)$\wedge$(q$\rightarrow$p))\\
			V & V & V & V & V & V\\
			V & F & F & F & V & F\\
			F & V & F & V & F & F\\
			F & F & V & V & V & V\\
		\end{tabular}
		(p\Iff q)$\equiv$((p$\rightarrow$q)$\wedge$(q$\rightarrow$p)) (Equivalencia material)
	\end{enumerate}
	
	\subsection{Ejercicio 9}
	
		\subsubsection{Inciso A}
		((p$\wedge$p)$\vee$p) $\equiv$ (Idempotencia de la conjunción)\\
		(p$\vee$p) $\equiv$ (Idempotencia de la disyunción)\\
		 p
		 
		 \subsubsection{Inciso B}
		 $\neg$($\neg$p$\vee$$\neg$q) $\equiv$ (Ley de De Morgan para la disyunción)\\
		 ($\neg$$\neg$p$\wedge$$\neg$$\neg$q) $\equiv$ (Doble negación)\\
		 (p $\wedge$ q)
	
		\subsubsection{Inciso C}
		( ( (p $\wedge$ ($\neg$p$\vee$q)) $\vee$ q ) $\vee$ (p $\wedge$ (p$\vee$q)) ) $\equiv$ (Distributiva para la conjunción)\\
		( ( ((p $\wedge$$\neg$p) $\vee$ (p$\wedge$q)) $\vee$ q) $\vee$ (p$\wedge$(p$\vee$q))) $\equiv$ (Inversa de la conjunción, Asociatividad de la conjunción, Absorción de la conjunción)\\
		( ( (False $\vee$ (p$\wedge$q) $\vee$ q) $\vee$ p)) $\equiv$ (Absorción de la disyunción)\\
		( ( False $\vee$ q ) $\vee$ p) $\equiv$ (Neutro de la disyunción)\\
		(q $\vee$ p)
		
		\subsubsection{Inciso D}
		($\neg$p $\rightarrow$ $\neg$(p$\rightarrow$$\neg$q)) $\equiv$ (Implicación Material)\\
		($\neg$p $\rightarrow$ $\neg$($\neg$p$\vee$$\neg$q)) $\equiv$ (Ley de De Morgan para la conjunción)\\
		($\neg$p $\rightarrow$ ($\neg$$\neg$p$\wedge$$\neg$$\neg$q)) $\equiv$ (Doble negación)\\
		($\neg$p $\rightarrow$ (p$\wedge$q)) $\equiv$ (Implicación Material)\\
		($\neg$$\neg$p $\vee$ (p$\wedge$q)) $\equiv$ (Doble Negación)\\
		(p $\vee$ (p$\wedge$q)) $\equiv$ (Absorción de la disyunción)\\
		p
		
		\subsubsection{Inciso E}
		(((p$\rightarrow$q) $\wedge$ (p$\wedge$$\neg$q))$\rightarrow$q) $\equiv$ (Implicación Material)\\
		((($\neg$p$\vee$q) $\wedge$ (p$\wedge$$\neg$q))$\rightarrow$q) $\equiv$ (Asociatividad de la conjunción, Distributividad de la conjunción)\\
		((((p$\wedge$$\neg$p) $\vee$ (p$\wedge$q)) $\wedge$ $\neg$q)$\rightarrow$q) $\equiv$ (Inversa de la conjunción)\\
		(((False $\vee$ (p$\wedge$q)) $\wedge$ $\neg$q)$\rightarrow$q) $\equiv$ (Neutro de la disyunción)\\
		(((p$\wedge$q) $\wedge$ $\neg$q)$\rightarrow$q) $\equiv$ (Asociatividad de la conjunción)\\
		((p $\wedge$ (q$\wedge$$\neg$q))$\rightarrow$q) $\equiv$ (Inversa de la conjunción)\\
		((p $\wedge$ False)$\rightarrow$q) $\equiv$ (Dominación de la conjunción)\\
		(False$\rightarrow$q) $\equiv$ (Implicación Material)\\
		($\neg$False $\vee$ q) $\equiv$ (True $\vee$ q) $\equiv$ (Dominación de la disyunción)\\
		True
		
		\subsubsection{Inciso F}
		$\neg$($\neg$(p $\wedge$ q) $\vee$ (p $\vee$ q))$\rightarrow$($\neg$$\neg$p $\vee$ $\neg$p) $\equiv$ (Doble negación, Ley de De Morgan para la conjunción)\\
		$\neg$(($\neg$p $\vee$ $\neg$q) $\vee$ p $\vee$ q)$\rightarrow$(p $\vee$ $\neg$p) $\equiv$ (Inversa de la disyunción, Asociatividad de la disyunción)\\
		$\neg$(($\neg$p $\vee$ p) $\vee$ ($\neg$q $\vee$ q))$\rightarrow$True $\equiv$ (Inversa de la disyunción)\\
		$\neg$(True $\vee$ True)$\rightarrow$True $\equiv$\\
		False $\rightarrow$ True $\equiv$\\
		True
	\subsection{Ejercicio 10}
	
		\subsubsection{Inciso A}
		((p$\wedge$p)$\wedge$p)$\rightarrow$p $\equiv$ (Idempotencia de la conjunción,Asociatividad de la conjunción)\\
		p$\rightarrow$p $\equiv$ (Implicación material)\\
		($\neg$ p $\vee$ p) $\equiv$ (Inversa disyunción)\\
		True\\
		
		Por lo tanto ((p$\wedge$p)$\wedge$p)$\rightarrow$p $\equiv$ True
		
		\subsubsection{Inciso B}
		(($\neg$p$\vee$$\neg$q)$\vee$(p$\wedge$q)$\rightarrow$(p$\wedge$q)) $\equiv$ (Asociatividad de la disyunción, Distributividad de la disyunción)\\
		(($\neg$p$\vee$$((\neg$q$\vee$p)$\wedge$($\neg$q$\wedge$q)))$\rightarrow$(p$\wedge$q)) $\equiv$ (Inversa de la disyunción)\\
		(($\neg$p$\vee$$((\neg$q$\vee$p)$\wedge$True))$\rightarrow$(p$\wedge$q)) $\equiv$ (Neutro de la conjunción)\\
		(($\neg$p$\vee$$(\neg$q$\vee$p))$\rightarrow$(p$\wedge$q)) $\equiv$ (Asociatividad de la disyunción)\\
		(($\neg$p$\vee$p)$\vee$$\neg$q)$\rightarrow$(p$\wedge$q)) $\equiv$ (Inversa de la disyunción)\\
		(True$\vee$$\neg$q)$\rightarrow$(p$\wedge$q)) $\equiv$ (Dominación de la disyunción)\\
		(True$\rightarrow$(p$\wedge$q)) $\equiv$ (Implicación material)\\
		($\neg$True$\vee$(p$\wedge$q)) $\equiv$ (False$\vee$(p$\wedge$q)) $\equiv$ (Neutro de disyunción)\\
		(p $\wedge$ q)\\
		
		Por lo tanto (($\neg$p$\vee$$\neg$q)$\vee$(p$\wedge$q)$\rightarrow$(p$\wedge$q)) $\equiv$ (p $\wedge$ q)
		
		\subsubsection{Inciso C}
		($\neg$p$\rightarrow$(q$\wedge$r)) $\equiv$ (Implicación material, Doble negación)\\
		(p$\vee$(q$\wedge$r)) $\equiv$ (Distributividad de la disyunción)\\
		((p$\vee$q)$\wedge$(p$\vee$r))\\
		
		Por lo tanto ($\neg$p$\rightarrow$(q$\wedge$r)) $\equiv$ ((p$\vee$q)$\wedge$(p$\vee$r))
		
		\subsubsection{Inciso D}
		$\neg$($\neg$p)$\rightarrow$($\neg$($\neg$p$\wedge$$\neg$q)) $\equiv$ (Ley de De Morgan para la conjunción, Doble negación)\\
		p$\rightarrow$(p $\vee$ q) $\equiv$ (Implicación material,Doble negación)\\
		($\neg$ p$\vee$(p$\vee$q)) $\equiv$ (Asociatividad de la disyunción, Inversa de la disyunción)\\
		(True $\wedge$ q) $\equiv$ (Dominación de la disyunción)\\
		True\\
		
		Por lo tanto $\neg$($\neg$p)$\rightarrow$($\neg$($\neg$p$\wedge$$\neg$q)) $\not\equiv$ q
		
		\subsubsection{Inciso E}
		((True $\wedge$ p)$\vee$($\neg$p$\vee$False)$\rightarrow$$\neg$($\neg$p$\vee$q)) $\equiv$ (Neutro de la conjunción, Neutro de la disyunción, Ley de De Morgan para la disyunción, Doble negación)\\
		((p$\vee$$\neg$p)$\rightarrow$(p$\wedge$$\neg$q)) $\equiv$ (Inversa de la conjunción)\\
		(False $\rightarrow$ (p $\wedge$$\neg$q)) $\equiv$ (Implicación Material)\\
		True $\vee$ (p $\wedge$$\neg$q)) $\equiv$ (Dominación de la disyunción)\\
		True\\
		
		Por lo tanto ((True $\wedge$ p)$\vee$($\neg$p$\vee$False)$\rightarrow$$\neg$($\neg$p$\vee$q)) $\not\equiv$ (p$\wedge$$\neg$q)
		
		\subsubsection{Inciso F}
		(p$\vee$($\neg$p$\wedge$q)) $\equiv$ (Distributiva de la disyunción)\\
		((p$\vee$$\neg$p)$\wedge$(p$\vee$q)) $\equiv$ (Inversa de la disyunción)\\
		(True$\wedge$(p$\vee$q)) $\equiv$ (Neutro de la conjunción)\\
		(p$\vee$q) $\equiv$ (Implicación Material)\\
		($\neg$p$\rightarrow$q)\\
		
		Por lo tanto, (p$\vee$($\neg$p$\wedge$q)) $\equiv$ ($\neg$p$\rightarrow$q)
		
		\subsubsection{Inciso G}
		$\neg$(p$\wedge$(q$\wedge$s)) $\equiv$ (Asociatividad de la conjunción)\\
		$\neg$(p$\wedge$q$\wedge$s) $\equiv$ (Ley de De Morgan para la conjunción)\\
		($\neg$p$\vee$$\neg$q$\vee$$\neg$s) $\equiv$ (Implicación Material)\\
		s$\rightarrow$($\neg$p$\vee$$\neg$q)\\
		
		Por lo tanto $\neg$(p$\wedge$(q$\wedge$s)) $\equiv$ s$\rightarrow$($\neg$p$\vee$$\neg$q)
		
		\subsubsection{Inciso H}
		(p$\rightarrow$(q$\wedge$$\neg$(q$\rightarrow$r))) $\equiv$ (Implicación Material)\\
		($\neg$p$\vee$(q$\wedge$$\neg$($\neg$q$\vee$r))) $\equiv$ (Ley de De Morgan para la disyunción, Doble negación)\\
		($\neg$p$\vee$(q$\wedge$(q$\wedge$$\neg$r))) $\equiv$ (Distributiva para la disyunción)\\
		(($\neg$p$\vee$q)$\wedge$($\neg$p$\vee$(q$\wedge$$\neg$r)))\\
		
		Por lo tanto, (p$\rightarrow$(q$\wedge$$\neg$(q$\rightarrow$r))) $\equiv$ (($\neg$p$\vee$q)$\wedge$($\neg$p$\vee$(q$\wedge$$\neg$r)))
		
	\subsection{Ejercicio 11}
		\subsubsection{Inciso A}
		Arranquemos probando que la conjunción es expresable mediante la negación y la disyunción. Propongo ver que (p$\wedge$q) $\equiv$ $\neg$($\neg$p$\vee$$\neg$q)\\
		\begin{tabular}{c|c|c|c|c|c|c}
			p & q & (p$\wedge$q) & $\neg$p & $\neg$q & ($\neg$p$\vee$$\neg$q) & $\neg$($\neg$p$\vee$$\neg$q)\\
			V & V & V & F & F & F & V\\
			V & f & F & F & V & V & F\\
			V & V & F & V & F & V & F\\
			V & V & F & V & V & V & F\\
		\end{tabular}
		
		\subsubsection{Inciso B}
		Como ya sabemos, (p$\rightarrow$q) $\equiv$ ($\neg$p $\vee$ q)  (Probado en 5.E)
		
		\subsubsection{Inciso C}
		Luego, como probamos que (p\Iff q) $\equiv$ ((p$\rightarrow$q)$\wedge$(q$\rightarrow$p)) en 8.U, y tanto ($\rightarrow$) como ($\wedge$) son expresables mediante negaciones y disyunciones, (\Iff)  también lo es
	\subsection{Ejercicio 12}
		\subsubsection{Inciso A}
		\begin{itemize}
			\item ($\emph{f}$ $\rightarrow$ $\neg$($\emph{e}$\Iff $\emph{m}$))
			\item ($\neg$$\emph{f}$ $\rightarrow$ $\neg$$\emph{e}$)
			\item (($\emph{e}$$\wedge$$\emph{f}$) $\rightarrow$ $\emph{m}$)
		\end{itemize}
		\subsubsection{Inciso B}
		Sabemos que si no es fin de semana, Juan no estudia (por el segundo item del Inciso A). Pero a la vez por el primer item, vemos que si es fin de semana, existe la posibilidad de que Juan estudie, siempre y cuando él no este escuchando música, ahora, el tercer item nos dice que si estudia y es fin de semana, escuchará música, lo cual haría incumplir el 1er item.
		Queda pendiente la verificación con lógica, pero creería que juntar las 3 proposiciones lleva a una contradicción.
	
	\subsection{Ejercicio 13}
		Defino $j = \text{Conocen a Juan; } c = \text{Conocen a Camila; } g = \text{Conocen a Gonzalo}$ \\
		
		\begin{tabular}{c|c|c|c|c|c|c|c}
			j & c & g & (j $\rightarrow$ c) & (c $\rightarrow$ g) & ((j $\rightarrow$ c)$\vee$(c $\rightarrow$ g)) & (j $\rightarrow$ g) & (((j $\rightarrow$ c)$\vee$(c $\rightarrow$ g))$\rightarrow$(j $\rightarrow$ g)) \\
			F          & F          & F          & V              & V              & V                          & V              & V                                      \\
			F          & F          & V          & V              & V              & V                          & V              & V                                      \\
			F          & V          & F          & V              & F              & F                          & V              & V                                      \\
			F          & V          & V          & V              & V              & V                          & V              & V                                      \\
			V          & F          & F          & F              & V              & F                          & F              & V                                      \\
			V          & F          & V          & F              & V              & F                          & V              & V                                      \\
			V          & V          & F          & V              & F              & F                          & F              & V                                      \\
			V          & V          & V          & V              & V              & V                          & V              & V                                     
		\end{tabular}\\
		Luego es cierto que si todos conocen a Juan entonces todos conocen a Gonzalo
	
	\subsection{Ejercicio 14}
		Si p = pelea y o = ojo morado. Luego se que $p \rightarrow o$ pero si o es verdadero, puede pasar que p sea True o que sea False indistintamente. Por ejemplo si digo "Si llueve, hay nubes negras", y luego afirmo que hay nubes negras, eso no quiere decir que esté lloviendo
		Esto es llamado falacia del consecuente
		
	\subsection{Ejercicio 15}
		\begin{enumerate}[label=(\alph*)]
			\item True
			\item True
			\item $\perp$
			\item $\perp$
			\item $\perp$
			\item $\perp$ 
		\end{enumerate}
	\subsection{Ejercicio 16}
		El operador $\implicaLuego$ se evalúa de forma secuencial, de izquierda a derecha, y es usado en lógica ternaria. Su tabla de verdad es:\\
		\begin{tabular}{c|c|c|c}
			p & q & p$\rightarrow$q & p$\implicaLuego$q\\
			V & V & V & V\\
			V & F & F & F\\
			F & V & V & V\\
			F & F & V & V\\
			V & $\perp$ & - & $\perp$\\
			F & $\perp$ & - & V\\
			$\perp$ & V & - & $\perp$\\
			$\perp$ & F & - & $\perp$\\
			$\perp$ & $\perp$ & - & $\perp$\\
		\end{tabular}
	\subsection{Ejercicio 17}
		El operador $\yLuego$ se evalúa de forma secuencial, de izquierda a derecha, y es usado en lógica ternaria. Su tabla de verdad es:\\
		\begin{tabular}{c|c|c|c}
			p & q & p$\wedge$q & p$\yLuego$q\\
			V & V & V & V\\
			V & F & F & F\\
			F & V & F & F\\
			F & F & F & F\\
			V & $\perp$ & - & $\perp$\\
			F & $\perp$ & - & F\\
			$\perp$ & V & - & $\perp$\\
			$\perp$ & F & - & $\perp$\\
			$\perp$ & $\perp$ & - & $\perp$\\
		\end{tabular}
	\subsection{Ejercicio 18}
		El operador $\oLuego$ se evalúa de forma secuencial, de izquierda a derecha, y es usado en lógica ternaria. Su tabla de verdad es:\\
		\begin{tabular}{c|c|c|c}
			p & q & p$\vee$q & p$\oLuego$q\\
			V & V & V & V\\
			V & F & V & V\\
			F & V & V & V\\
			F & F & F & F\\
			V & $\perp$ & - & V\\
			F & $\perp$ & - & $\perp$\\
			$\perp$ & V & - & $\perp$\\
			$\perp$ & F & - & $\perp$\\
			$\perp$ & $\perp$ & - & $\perp$\\
		\end{tabular}
	\subsection{Ejercicio 19}
		$b=c=True, a=False, x=y=\perp$
		\begin{enumerate}[label=(\alph*)]
		\item $(\neg x \oLuego b) \equiv (\neg \perp \oLuego True) \equiv \perp$
		\item $((c \yLuego (y \yLuego a))\oLuego b) \equiv (True \oLuego(\perp \yLuego False)\oLuego True) \equiv (True \oLuego \perp \oLuego True) \equiv True$
		\item $\neg (c \oLuego y) \equiv \neg (True \oLuego \perp) \equiv \neg True \equiv True$
		\item $(\neg(True \oLuego \perp)\Iff(\neg True \yLuego \neg \perp)) \equiv (\neg True \Iff False) \equiv (False \Iff False) \equiv True$
		\item $((True \oLuego \perp) \yLuego (False \oLuego True)) \equiv (True \yLuego True) \equiv True$
		\item $(True \Iff True) \equiv True$
		\item $(\neg True \yLuego \neg \perp) \equiv (False \yLuego \perp) \equiv False$
		\end{enumerate}
	\subsection{Ejercicio 20}
		$p=True, q=False, r=\perp$
		\begin{enumerate}[label=(\alph*)]
		\item $(True \yLuego True) \equiv True$
		\item $(False \implicaLuego (True \yLuego False)) \equiv (False \implicaLuego False) \equiv True$
		\item $(True \implicaLuego (True \oLuego \perp)) \equiv (True \implicaLuego True)  \equiv True$
		\item $(False \oLuego (\perp \yLuego (False \yLuego True))) \equiv (False \oLuego (\perp \yLuego False)) \equiv (False \oLuego \perp) \equiv \perp$
		\end{enumerate}
\end{document}
